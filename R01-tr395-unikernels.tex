\documentclass[11pt]{article}

\usepackage{a4wide, parskip}
\usepackage{cite}
\usepackage{hyperref}

\begin{document}

%% Replace PAPERTITLE below with the paper title
\title{Paper Review Form (1000 words maximum)\\
    tr395: Unikernels: Library Operating Systems for the Cloud \cite{Unikernels}}

\maketitle

\section*{Paper Summary}

\textsl{3--5 sentences. Briefly summarise the {\bf contributions} of the paper,
i.e.,~what it adds over the state of the art. Paraphrase and extract the
essentials rather than simply copying chunks of text. Be objective; later
sections allow for your own opinion.}

The paper presents the idea of unikernels, single-purpose applications that use
a \textit{library operating system} to be compiled down into standalone
kernels. Additionally, an OCaml implementation of this idea called MirageOS is
provided and analysed. This improves over the state of the art in taking
advantage of the abstractions hypervisors provide to avoid the hardware
compatibility issues that previous library operating systems have fallen
victim to. The lightweight nature of MirageOS also means that it is a lot less
heavyweight than traditional operating systems.

\section*{Pros and Cons}

\textsl{6 bullets. Succinctly state three positives and three negatives of the
paper.}

Pros:

\begin{itemize}

    \item The idea of using a type-safe language at a systems-level could
    provide a lot of potential benefits in the long term, despite the short
    term costs.

    \item The paper is extremely comprehensive, considering many of the pros
    and cons of the ideas and implementation presented.

    \item The paper presents a deep exploration of a novel idea that if
    successful, needs initial research like this to investigate its usefulness.

    %\item Concrete example of benefits given in DNS server.

\end{itemize}

Cons:

\begin{itemize}

    % \item XeN is falling out of favour (TODO: reference).

    %    https://www.theregister.co.uk/2017/11/07/aws_writes_new_kvm_based_hypervisor_to_make_its_cloud_go_faster/


    % \item The world has moved on from VMs to containers

    \item There is so much existing infrastructure which depends on the deep
    feature-sets Operating Systems provide, one of the reasons this project is
    so lightweight is because it is lacking a lot of those features.

    \item OCaml, despite its features, isn't a pragmatic language choice as it
    is relatively esoteric and not widely understood. (This is acknowledged in
    the paper).

    % \item 'minor' setbacks in the performance graphs mean that the performance
    % impact at scale is relatively huge.

    \item Some of the evaluation section seems arbitrary, for example showing
    the time to construct millions of threads in parallel where each thread
    sleeps for between 0.5–1.5 seconds and then terminates.

    % \item The sytem isn't backwards compatible. (acknowledged in the paper)

    % \item In the ping test, 4-10\% increase in latency compared to Linux is
    % considered \textit{small}.

    % \item So many, many edge-cases that would need to be considered.
    % Redevlopment cost of linux kernel estimated at 744GBP. %https://en.wikipedia.org/wiki/Linux_kernel#Estimated_cost_to_redevelop

    % \item benchmarks are arbitrary

    % \item no nice overall overview of the evaluation (contents or conclusion)

    % \item Line of code comparisons are arbitrary.

    % \item mentions neat benefits of being compiled to nodejs

    % \item performance impact downplayed

    % \item parallelisation depends on multiple VMs

    % \item a lot of infrastructure needs to be rebuilt in Mirage, but the real
    % world will keep moving forward.

    % \item LoC Count

\end{itemize}


\section*{The Problem/Motivation}

\textsl{1--2 sentences per question. What is the motivation for the work, or
the problem being solved? Why is it important? If there is prior art, how was
it insufficient? If the problem had not previously been solved, why not?}


\subsection*{Problem motivation}

The way internet services have been used and deployed has evolved and developed
significantly over the past few years, mainly with the growing use of
hypervisor and many heavyweight Virtual Machines. However, spawning complete
instances operating systems (such as GNU/Linux, or Windows) which then execute
a single task (eg. a web server, or a database) could be considered wasteful.


\subsection*{Importance of problem}

The problem is important because of the scale at which these systems operate,
with estimates that Amazon's EC2 is made up of almost 500,000 servers
\cite{EC2Amount}. Any kind of resource savings in computation, power or memory
(however minor for individual instances), can have major consequences at scale.

\subsection*{Why the problem hasn't been previously solved}

The problem is one where incremental improvements can always be possible and
beneficial. Furthermore, due to the rapidly advancing nature of the systems in
question, optimisations will be possible for the foreseeable future. There is
also the question of tradeoffs, where the number of variables in play mean that
some solutions will always have some advantages compared to others based on the
decisions they make. This paper explores a single previously unexplored
avenue.


\section*{The Solution/Approach}

\textsl{5--10 sentences. What have they done? How does it address the issues
set out above? How is it unique and/or innovative (if, indeed, it is)? Give
details, again using the paper as the source but again, not just copying text.
Instead, focus on paraphrasing/synopsising, and extracting the essential
details.} %% Type your text below

The main thing done to address the issues set out above is the introduction and
development of an idea known as the unikernel. This is a library operating
system that can be used to develop applications, which then get compiled down
to a kernel that can run on a hypervisor. This is innovative because previous
attempts at developing library operating systems have had limited success due
to the difficulty of supporting a wide range of hardware, which hypervisors can
abstract away. As the example unikernel in the paper, \textit{MirageOS}, was
written in the functional programming language OCaml, backwards compatibility
with existing systems is sacrificed in exchange for a lot of potential security
benefits due to the type safety of such a language.

%TODO: Expand on this with more details

\section*{Evaluation}

\textsl{3--4 sentences. How do they evaluate their work? What questions does
their evaluation set out to answer? What does their evaluation say about the
strengths and weaknesses of their system? What is your opinion of the strengths
and weaknesses of the evaluation itself? Give highlights, not a point by point
reproduction of the evaluation section(s). In rare cases, systems papers may
not have any evaluation, in which case write `N/A' below.}

The paper performs an evaluation in two forms - through benchmarks and LoC
(line of code) counts. There are a few different types of benchmarks; the first
being \textit{microbenchmarks}, by examining the performance of specific
features of MirageOS and comparing that with the performance of those features
in conventional operating systems. Elements such as boot-time, threading,
networking and storage are analysed for the microbenchmarks. Various examples
of potential real-world applications are then evaluated such as a DNS Server,
an OpenFlow controller and a dynamic web server that implements a
\textit{Twitter-like} service. Most predictable but notable result is the size
of applications - with Linux versions coming in at 100s of MB compared to the
100s of KBs for MirageOS applications. Other gains include memory size and boot
time.

The question the evaluation essentially aims to set out to answer is: How does
MirageOS compare to conventional operating systems?

The weakest part of the evaluation in my opinion is the section on code size,
despite attempts to normalize the results, the results themselves aren't really
insightful.

\section*{Your Opinion}

\textsl{At least 3 sentences. This is the fun part where you get to judge both
the paper and the work it reports! Is the motivation convincing? The problem
important? The approach a good or bad idea? Why? Which specific things annoyed
you, or you thought were cool, or cool-but-flawed? Justify your opinions! Make
an argument which will convince others of your opinion.}

Overall, I think the work reported by the paper is extremely cool. Although
before it can be used at a large scale, a lot of work needs to be done. Due to
the sheer amount of existing infrastructure I think something like this should
initially live alongside current VMs and be used for specialised purposes.
Having checked the project web page \cite{MirageOSWeb}, development on the
project seems to be active which is encouraging.

The paper itself is extremely thorough and the arguments presented are
convincing. Additionally, The potential drawbacks and pitfalls of the project
are noted and explored. The main issue I have with the paper is the evaluation
section. Some of the benchmarks seem arbitrary (such as waking and sleeping
many threads), where even when they might be justified, the actual reasons for
the choices of benchmarks are not given. Not much justification is given as to
why line-of-code count is an important metric to evaluate. Finally, there are
random asides such as a mention that MirageOS could be compiled to nodejs,
without stating why that would be beneficial.

\section*{Questions for the Authors}

\textsl{Finally, imagine you're attending a talk about this paper given by one
of the authors. Give at least 2 questions that you would like to ask, specific
to the paper and the research it reports.}

If taking a clean-slate approach to this project, or the idea of unikernels,
what would you change based on what you learnt?

What kind of software/projects can you think of where you would recommend
people use MirageOS over other options considering its current state?

\bibliographystyle{unsrt}

\bibliography{references}

\end{document}
