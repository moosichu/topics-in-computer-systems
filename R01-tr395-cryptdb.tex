\documentclass[11pt]{article}

\usepackage{a4wide, parskip}
\usepackage{cite}
\usepackage{hyperref}

\begin{document}

% This work is licensed under the Creative Commons Attribution-NoDerivatives 4.0
% International License. To view a copy of this license, visit
% http://creativecommons.org/licenses/by-nd/4.0/ or send a letter to Creative
% Commons, PO Box 1866, Mountain View, CA 94042, USA.

%% Replace PAPERTITLE below with the paper title
\title{Paper Review Form (1000 words maximum)\\
    tr395: CryptDB: Protecting Confidentiality with Encrypted Query Processing \cite{CryptDB}}
\maketitle

\section*{Paper Summary}

\textsl{3--5 sentences. Briefly summarise the {\bf contributions} of the paper,
i.e.,~what it adds over the state of the art. Paraphrase and extract the
essentials rather than simply copying chunks of text. Be objective; later
sections allow for your own opinion.}

The paper presents CryptDB, a system for encrypting data stored in SQL
databases, such that administrators can never access the decrypted data and to
ensure that adversaries can only decrypt the data of users who are logged in if
they manage to compromise the system. CryptDB works using a combination of
three ideas: executing SQL queries over encrypted data by using a
\textit{SQL-aware encryption strategy}, using an \textit{adjustable query-based
encryption scheme} to account for the fact that some queries may require
encryption schemes that look more information and others, and finally
\textit{chaining encryption keys to user passwords} so that data can only be
decrypted when a given user is logged in. Although these techniques build on
existing work, this is the first system which operates on top-of commercial
database-management systems and applies the techniques in a comprehensive
fashion.

\section*{Pros and Cons}


\textsl{6 bullets. Succinctly state three positives and three negatives of the
paper.}

Pros:

\begin{itemize}

    \item CryptDB doesn't require any modifications to the underlying DBMS
    systems - it simply acts as a proxy server making adoption more
    straight-forward.

    \item It can integrate and interface nicely with existing applications,
    requiring only 2-7 lines of code changes for three example applications in
    the paper.

    \item Encrypting data such that in-the-face-of compromise, it is still
    unreadable is a pragmatic approach to security.

\end{itemize}

Cons:

\begin{itemize}

    \item The paper fundamentally confuses two ideas. Firstly, the onion-based
    encryption of columns using a master key stored by CryptDB to allow each
    column to reach a steady-state of minimumly-acceptable security. Secondly,
    the select per-data-item encryption used to secure data at a per-user level
    such that data is confidential in the face of compromise when the user
    isn't \textit{logged in}.

    % \item Claims to have low overhead, despite reducing throughput by 26\% in
    % one case and requiring over triple the storage requirements for a database
    % in another case!

    % \item Many applications such as TCP-C, PHPbb, Hot-CRP and Grad-apply are
    % mentioned without reference.

    % \item the TCP-C benchmark doesn't seem applicable without correct security
    % configurations.

    \item In order to provide meaningful security improvements, there is an
    implicit assumption that the number of logged in users at any one time will
    be sparse.

    % \item Does place restrictions on the DBMS, and potential reduces
    % maintainability for administrators.

    \item The default behaviour of lazily peeling-back onion layers when needed
    without question means that the steady state of any system will also be the
    least-secure state of that system. Even if a given query is only executed
    once in the lifetime of it.

    % \item No real definition of \textit{user} given until later.

    % \item The fact that the current set of logged-in users is handled by a
    % magic SQL table that is actually on the proxy is really odd.

    % \item What if the application fails to log users out? Steady state for this
    % seems to be the least-secure.

    % \item Freely interchanges security and confidentially as terms, when they
    % are different.

    % \item There seem to be two different ideas in play. \textit{sensitive}
    % information seems different to regularly-encrypted information (which is
    % encrypted by master-key), which has the same problems as before.

    % \item The SEARCH encryption scheme requires the user to know what they are
    % going to need to search for in the first place.

    % \item The example code given is horrendously formatted. (Different
    % non-mono-spaced fonts are used, alongside italics!)

    % \item Large chains of \textit{speaks-for} relationships could create a
    % weakest-link system (the security of the system depends who is logged on at
    % any given point in time!).

    % \item No consideration is given for how non-password based systems could be
    % handled (such as single sign-on, or magic email-links).

    % \item No mention on how whether the proxy supports prepared statements (is
    % it prone to SQL-injections?)

    % \item Some statements made, such as the fact that CryptDB should be run on
    % a different physical machine from the DBMS, without saying \textit{why}.

    % \item No mention of error handling (what if an operation isn't possible?).

    % \item Delayering the onion could cause massive stalls.

    % \item The DBMS performs the decryption of a given onion-layer itself when
    % needed, being given a copy of the private key for said layer! That could be
    % logged in the DBMS system!

    % \item What happens if the master key is compromised?

    % \item Nothing is mentioned about migrating to CryptDB from a non-CryptDB
    % database.

\end{itemize}

\section*{The Problem/Motivation}

\textsl{1--2 sentences per question. What is the motivation for the work, or
the problem being solved? Why is it important? If there is prior art, how was
it insufficient? If the problem had not previously been solved, why not?}

The motivation for the work is the fact that all systems will inevitably be
insecure on some level, including database-management systems (DBMS).
Therefore, there is pragmatic value in systems which allows for some-measure of
confidentiality even when an adversary gains complete access to a system. A
system which can increase the confidentiality of DBMSs in the manner described
potentially has many use-cases.. There is plenty of prior-art in increasing the
confidentiality of DBMSs, including on-disk encryption, but many of these are
insufficient in requiring the decryption of data for basic queries and relying
on master-keys which can themselves be used to decrypt data when the system is
compromised.

\section*{The Solution/Approach}

\textsl{5--10 sentences. What have they done? How does it address the issues
set out above? How is it unique and/or innovative (if, indeed, it is)? Give
details, again using the paper as the source but again, not just copying text.
Instead, focus on paraphrasing/synopsising, and extracting the essential
details.}

The authors of the paper developed the a system called CryptDB which provides
confidentiality in two forms, each using a different technique. For both
techniques, the goal is to intercept SQL queries from an application, and
convert them into an encrypted form such that as many operations in a DBMS as
possible operate on-and-using encrypted data.

The first technique essentially provides \textit{minimally-acceptable}
encryption of data in the general case. Using a master-key, by-default CryptDB
encrypts all data as securely as possible using various layers of encryption.
However, this level of encryption doesn't for allow certain SQL operations
(such as \texttt{ORDER BY}), so when these operations are used CryptDB will
decrypt the layers of encryption until the data is encrypted in a form where
the required information for those operations is leaked. As the decryption
happens lazily, a steady-state of \textit{minimally-acceptable} encryption is
reached. The innovation comes the combination of existing and new techniques to
allow for the encryption to happen relatively transparently from both an
application's and a DBMS's perspective.

The second technique allows certain specified sensitive information to be
encrypted such that only the users who have permission to access the data may
decrypt it. Therefore an adversary can only access such data if a user with
relevant permission to a system is logged-in.


\section*{Evaluation}

\textsl{3--4 sentences. How do they evaluate their work? What questions does
their evaluation set out to answer? What does their evaluation say about the
strengths and weaknesses of their system? What is your opinion of the strengths
and weaknesses of the evaluation itself? Give highlights, not a point by point
reproduction of the evaluation section(s). In rare cases, systems papers may
not have any evaluation, in which case write `N/A' below.}

The evaluation section consists of four parts - a brief discussion on
usability, a functional evaluation, a security evaluation and a performance
evaluation.

The functional evaluation indicates that a strength of the system comes from
the minimal changes applications need in order to use CryptDB en-lieu of a
plaintext DBMS. Furthermore, the system is capable of supporting a large
proportion of queries from a fairly comprehensive trace. However, even if 99\%
of queries used are supported \textit{out-of-the-box}, the question is how
important the last 1\% is.

The security evaluation simply seeks to analyse which levels of encryption
different columns reach in the steady-state of typical usage. Unfortunately,
the table used to give an overview of the steady-state of different
applications features raw numbers, as opposed to percentages which would have
made comparing the different rows more straight-forward.

Finally, there is the performance evaluation, which seeks to answer what the
overhead of using CryptDB with a MySQL server is. The impact varies from
7\%-100\% for throughput depending on the context and benchmark, to even an
over 200\% impact on storage requirements in some situations. How acceptable
the overhead is depends entirely on the context within which CryptDB would be
desirable to use.

The performance evaluation suffers from many of the issues common to systems
papers - notably the lack of any kind of statistical analysis and the lack of
running multiple trials of each benchmark. Furthermore, I was surprised to find
that there was no evaluation of the \textit{cost} of the cryptographic features
provided by CryptDB - what is the overhead like when CryptDB is used without
those? That question is arguably an important one for migrations to CryptDB.

\section*{Your Opinion}

\textsl{At least 3 sentences. This is the fun part where you get to judge both
the paper and the work it reports! Is the motivation convincing? The problem
important? The approach a good or bad idea? Why? Which specific things annoyed
you, or you thought were cool, or cool-but-flawed? Justify your opinions! Make
an argument which will convince others of your opinion.}

I think both the paper and the work it reports provide a lot of value and
potential, and that the ideas presented are both pragmatic and an improvement
of the status quo. However, the paper isn't without issues, as I wrote down 23
significant points of concern over course of reading the paper. These issues
ultimately all stem from the fact that the paper leaves a lot of the potential
questions about CryptDB unanswered, from the relatively simple (Does the system
support prepared statements? What if an adversary blocks the specific traffic
from an application to CryptDB that logs users out? Can't the master key for
the onion-encryption be compromised?), to higher-level (How is migration to
CryptDB handled? What challenges do you see CryptDB facing?). There are also
simpler issues, such as the introduction of terms such as \textit{users}
without clarification until late in the paper, and poor formatting of example
code.

\section*{Questions for the Authors}

\textsl{Finally, imagine you're attending a talk about this paper given by one
of the authors. Give at least 2 questions that you would like to ask, specific
to the paper and the research it reports.}

For managing logged-in users, why was the decision made to have CryptDB spoof
the existence of a \textit{magic} SQL table, instead of providing a bespoke
API?

What do you see as the biggest challenges for the potential adoption of
CryptDB?

\bibliographystyle{unsrt}

\bibliography{references}

\end{document}
