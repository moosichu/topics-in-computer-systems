\documentclass[11pt]{article}

\usepackage{a4wide, parskip}
\usepackage{cite}
\usepackage{hyperref}

\begin{document}

% This work is licensed under the Creative Commons Attribution-NoDerivatives 4.0
% International License. To view a copy of this license, visit
% http://creativecommons.org/licenses/by-nd/4.0/ or send a letter to Creative
% Commons, PO Box 1866, Mountain View, CA 94042, USA.

%% Replace PAPERTITLE below with the paper title
\title{Paper Review Form (1000 words maximum)\\
    tr395: CryptDB: Protecting Confidentiality with Encrypted Query Processing \cite{CryptDB}}
\maketitle

\section*{Paper Summary}

\textsl{3--5 sentences. Briefly summarise the {\bf contributions} of the paper,
i.e.,~what it adds over the state of the art. Paraphrase and extract the
essentials rather than simply copying chunks of text. Be objective; later
sections allow for your own opinion.}

The paper presents CryptDB, a system for encrypting data stored in SQL
databases, such that administrators can never access the decrypted data and to
ensure that adversaries can only decrypt the data of users who are logged in if
they manage to compromise the system. CryptDB works using a combination of
three ideas: executing SQL queries over encrypted data by using a
\textit{SQL-aware encryption strategy}, using an \textit{adjustable query-based
encryption scheme} to account for the fact that some queries may require
encryption schemes that look more information and others, and finally
\textit{chaining encryption keys to user passwords} so that data can only be
decrypted when a given user is logged in. Although these techniques build on
existing work, such as cryptographic tools for performing keyword search over
encrypted data (TODO: cite song et al.), this is the first system which can
work on top-of commercial database-management systems such as MySQL and
PostreSQL (TODO: cite mysql and postresql) and applies the techniques in such a
comprehensive fashion.

\section*{Pros and Cons}


\textsl{6 bullets. Succinctly state three positives and three negatives of the
paper.}

Pros:

\begin{itemize}

    \item CryptDB doesn't require any modifications to the underlying DBMS
    systems - it simply acts as a proxy server making adoption more
    straight-forward.

    \item It can integrate and interface nicely with existing applications,
    requiring only 2-7 lines of code changes for three example applications in
    the paper.

    \item Encrypting data such that in-the-face-of compromise, it is still
    unreadable is very pragmatic approach to security.

\end{itemize}

Cons:

\begin{itemize}

    \item The paper fundamentally confuses two ideas. Firstly, the onion-based
    encryption of all columns using a master key stored by CryptDB to allow
    each column to reach a steady-state of minimumly-acceptable security.
    Secondly, the per-data-item state of encryption used to secure data at a
    per-user level such that data is confidential in the face of compromise
    when the user isn't \textit{logged in}.

    \item Claims to have low overhead, despite reducing throughput by 26\% in
    one case and requiring over triple the storage requirements for a database
    in another case!

    \item Many applications such as TCP-C, PHPbb, Hot-CRP and Grad-apply are
    mentioned without reference.

    \item the TCP-C benchmark doesn't seem applicable without correct security
    configurations.

    \item In order to provide meaningful security improvements, there is an
    implicit assumption that the number of logged in users at any one time will
    be sparse.

    \item Does place restrictions on the DBMS, and potential reduces
    maintainability for administrators.

    \item The default behaviour of lazily peeling-back onion layers when needed
    without question means that the steady state of any system will also be the
    least-secure state of that system. Even if a given query is only executed
    once in the lifetime of the system.

    \item No real definition of \textit{user} given until later.

    \item The fact that the current set of logged-in users is handled by a
    magic SQL table that is actually on the proxy is really odd.

    \item What if the application fails to log users out? Steady state for this
    seems to be the least-secure.

    \item Freely interchanges security and confidentially as terms, when they
    are different.

    \item There seem to be two different ideas in play. \textit{sensitive}
    information seems different to regularly-encrypted information (which is
    encrypted by master-key), which has the same problems as before.

    \item The SEARCH encryption scheme requires the user to know what they are
    going to need to search in the first place.

    \item The example code given is horrendously formatted. (Different
    non-mono-spaced fonts are used, alongside italics!)

    \item Large chains of \textit{speaks-for} relationships could create a
    weakest-link system (the security of the system depends who is logged on at
    any given point in time!).

    \item No consideration is given for how non-password based systems could be
    handled (such as single sign-on, or magic email-links).

    \item No mention on how whether the proxy supports prepared statements (is
    it prone to SQL-injections?)

    \item Some statements made, such as the fact that CryptDB should be run on
    a different physical machine from the DBMS, without saying \textit{why}.

    \item No mention of error handling (what if an operation isn't possible?).

    \item Delayering the onion could cause massive stalls.

    \item The DBMS performs the decryption of a given onion-layer itself when
    needed, being given a copy of the private key for said layer! That could be
    logged in the DBMS system!

    \item What happens of the master key is compromised?

\end{itemize}

\section*{The Problem/Motivation}

\textsl{1--2 sentences per question. What is the motivation for the work, or
the problem being solved? Why is it important? If there is prior art, how was
it insufficient? If the problem had not previously been solved, why not?}



\section*{The Solution/Approach}

\textsl{5--10 sentences. What have they done? How does it address the issues
set out above? How is it unique and/or innovative (if, indeed, it is)? Give
details, again using the paper as the source but again, not just copying text.
Instead, focus on paraphrasing/synopsising, and extracting the essential
details.}


\section*{Evaluation}

\textsl{3--4 sentences. How do they evaluate their work? What questions does
their evaluation set out to answer? What does their evaluation say about the
strengths and weaknesses of their system? What is your opinion of the strengths
and weaknesses of the evaluation itself? Give highlights, not a point by point
reproduction of the evaluation section(s). In rare cases, systems papers may
not have any evaluation, in which case write `N/A' below.}

\item No control evalation (usage of system with all encryption turned off?)

\item Evlauton had 10 users???

\item Evaluation suffers from typical issues (not multiple trials, etc.)


\section*{Your Opinion}

\textsl{At least 3 sentences. This is the fun part where you get to judge both
the paper and the work it reports! Is the motivation convincing? The problem
important? The approach a good or bad idea? Why? Which specific things annoyed
you, or you thought were cool, or cool-but-flawed? Justify your opinions! Make
an argument which will convince others of your opinion.}
below


\section*{Questions for the Authors}

\textsl{Finally, imagine you're attending a talk about this paper given by one
of the authors. Give at least 2 questions that you would like to ask, specific
to the paper and the research it reports.}

\bibliographystyle{unsrt}

\bibliography{references}

\end{document}
