\documentclass[11pt]{article}

\usepackage{a4wide, parskip}
\usepackage{cite}
\usepackage{hyperref}

\begin{document}

% This work is licensed under the Creative Commons Attribution-NoDerivatives 4.0
% International License. To view a copy of this license, visit
% http://creativecommons.org/licenses/by-nd/4.0/ or send a letter to Creative
% Commons, PO Box 1866, Mountain View, CA 94042, USA.

%% Replace PAPERTITLE below with the paper title
\title{Paper Review Form (1000 words maximum)\\
    tr395: Pivot Tracing: Dynamic Causal Monitoring for Distributed Systems \cite{PivotTracing}}

\maketitle

\section*{Paper Summary}

\textsl{3--5 sentences. Briefly summarise the {\bf contributions} of the paper,
i.e.,~what it adds over the state of the art. Paraphrase and extract the
essentials rather than simply copying chunks of text. Be objective; later
sections allow for your own opinion.}

The paper presents Pivot Tracing, a system for monitoring framework for
distributed systems. It allows for dynamic instrumentation similar to what
systems like DTrace provide \cite{DTrace}, with the contribution that traces
can be propagated across system boundaries. This is implemented using a novel
\textit{happened-before join} operator, which allows SQL-like join operations
to be used to aggregate data across systems.

\section*{Pros and Cons}

\textsl{6 bullets. Succinctly state three positives and three negatives of the
paper.}

Pros:

\begin{itemize}

    % \item Having tracepoints only be references until at given query installed
    % is a good idea that helps reduce overhead.

    \item The motivation is clear and well-presented, allowing the reader to
    fully understand the context of the problem going into the paper.

    \item Although not part of the paper itself, a filmed presentation of it is
    easily available and complements it nicely.

    \item The discussion section demonstrates some good ideas and presents
    possible avenues for further work, it also helps frame the authors'
    perspective of the system and helps answer questions that may spring to a
    reader's mind (such as, is the system Java specific?).

\end{itemize}

Cons:

\begin{itemize}

    % \item The overhead measurement section in the evaluation is really brief
    % and lacking possible test cases.


    \item Requires fairly significant modifications to the source code of what
    is being traced. (50-200 lines of code need to be implemented in order to
    support the 'baggage' system?)

    % \item Tracepoints have to be manually defined. Application level

    % \item Implements Baggage. (key-value container)

    % \item Any mentino of the probe effect?

    % \item Tracebo

    % \item No common theme in naming is actually confusing (Advice, Baggage,
    % Weaving).

    % \item Spends a lot of time on the bug found in Hadoop.

    % \item Benchmarks have a 0.3\% baseline overhead, but do they test for
    % potential worst cases?

    % \item Maximum overhead of 14.3\%.

    % \item Although per-workload overhead is 1.5\%, what is the global overhead?

    \item Even after watching the video and feeling like I have a good
    understanding of the system, Figure 6 (on how the optimised query works)
    still feels unhelpful to that understanding.

    \item The \textit{happened-before join} operator is an abstract
    representation of a fairly straight-forward idea - the focus on that
    arguably makes the paper harder to understand.

\end{itemize}

\section*{The Problem/Motivation}

\textsl{1--2 sentences per question. What is the motivation for the work, or
the problem being solved? Why is it important? If there is prior art, how was
it insufficient? If the problem had not previously been solved, why not?}

Instrumentation is an important part of maintaining systems
\cite{InstrumentationImportance}. Unfortunately, previous techniques lack
desirable features, such as being able to arbitrarily decide instrumentation at
run time (as is the case with logs), or being able to instrument code across
system boundaries such as machines (as is the case with DTrace \cite{DTrace}.
The reason this problem hasn't been previously solved is that being able to
arbitrarily instrument code across boundaries in a distributed system is highly
challenging.


\section*{The Solution/Approach}

\textsl{5--10 sentences. What have they done? How does it address the issues
set out above? How is it unique and/or innovative (if, indeed, it is)? Give
details, again using the paper as the source but again, not just copying text.
Instead, focus on paraphrasing/synopsising, and extracting the essential
details.}

The authors have introduced a technique called Pivot Tracing, that allows for
the dynamic instrumentation DTrace and Fay provide \cite{DTrace}, in addition
to allowing users to correlate information from different points of the system
as found in causal tracing. The innovation the paper claims to make is in the
introduction of a \textit{happened-before join} operator, which is based on
Lamport's happened-before relation. The paper additionally presents a
Java-based implementation of this system, that can run on various Apache Hadoop
\cite{Hadoop} services. This implementation consists of an agent-based system
that runs a thread inside each process that can dynamically instrument Java, in
addition to a \textit{Baggage} library for use by what is being monitored.
Finally, a front-end client library is used to define tracepoints, compile
queries and submit the results of this to the agents.

The system works by allowing SQL-like statements to be compiled to an
\textit{Advice} format, where separate components are distributed to different
tracepoints in the system. These then allow desired parameters to be bundled
into tuples called \textit{Baggage}, which are passed along the execution path
in the distributed system. Desired operations for aggregating information are
performed local to the tracepoints where possible in as an optimisation.


\section*{Evaluation}

\textsl{3--4 sentences. How do they evaluate their work? What questions does
their evaluation set out to answer? What does their evaluation say about the
strengths and weaknesses of their system? What is your opinion of the strengths
and weaknesses of the evaluation itself? Give highlights, not a point by point
reproduction of the evaluation section(s). In rare cases, systems papers may
not have any evaluation, in which case write `N/A' below.}

The evaluation of the Pivot Tracing prototype consists of 3 sections: a case
study, a demonstration of its end-to-end latency recording features, and an
analysis of the overhead the system presents. The evaluation aims to answer:
can Pivot Tracing help solve real-world problems, and what overhead does the
system present? The former question is answered by a thorough breakdown on how
a bug in Hadoop was first noticed, and then tracked-down to its source using
the Pivot Tracing. The second section expands on this by showing how the
latency-measuring features of the system allowed a faulty-link to be
discovered. The final section aims to measure the performance overhead of the
Pivot Tracing system, by showing the latency effects of increased amounts of
\textit{baggage} in a micro-benchmark, before demonstrating the latency
overheads of different configurations of Pivot Tracing in a single
marco-benchmark.

In my opinion whilst the rest of the evaluation demonstrates what it aims to do
quite thoroughly, the benchmarking section is lacking in ways both common to
systems papers (a lack of multiple trials, and statistical analysis of the
results), and in ways unique to this paper. The first oddity is a lack of any
kind of throughput overhead analysis - the focus is entirely on latency. Secondly,
although the paper mentions the fact that four separate benchmark were tested,
only the results from one were displayed.

There is also a lack of analysis of any probe-effect that Pivot Tracing could
incur on a system. Finally, I would have liked to have seen a comparison of the
performance of Pivot Tracing statements that are semantically equivalent to the
traces Hadoop itself provides.


\section*{Your Opinion}

\textsl{At least 3 sentences. This is the fun part where you get to judge both
the paper and the work it reports! Is the motivation convincing? The problem
important? The approach a good or bad idea? Why? Which specific things annoyed
you, or you thought were cool, or cool-but-flawed? Justify your opinions! Make
an argument which will convince others of your opinion.}

Overall I think the paper was well presented and understandable, although a
video of the presentation on it helped greatly with that. The figures
especially seemed designed around the animations the presentation could
display, and were harder to parse in the paper itself. Due to my lack of domain
expertise with using distributed systems, it is hard to evaluate the
\textit{usefulness} of a system like this and whether the overheads it presents
are worth the benefits it provides. My largest grievance with the approach is
the fairly significant changes that had to be made to each module in the Hadoop
framework of 50~100 lines of code. However, if this were merged upstream that
would be much less of an issue, and in theory an improvement over what these
systems already aim to provide.

Finally, the evaluation of the system is far too focused on the case-study
(over multiple pages), compared to the half-page the performance evaluation
receives.


\section*{Questions for the Authors}

\textsl{Finally, imagine you're attending a talk about this paper given by one
of the authors. Give at least 2 questions that you would like to ask, specific
to the paper and the research it reports.}

Is the eventual goal to see tracing systems like this merged into the upstream
of distributed systems like \textit{Hadoop}, or for a standard to emerge that
can live externally to unmodified versions of them?

Although the \textit{happens-before join} is the fundamental operation around
which this system is built, are there any others that you think could be
beneficial to a system like this?

\bibliographystyle{unsrt}

\bibliography{references}

\end{document}
