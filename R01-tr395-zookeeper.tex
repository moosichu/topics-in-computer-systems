\documentclass[11pt]{article}

\usepackage{a4wide, parskip}
\usepackage{cite}
\usepackage{hyperref}

\begin{document}

% This work is licensed under the Creative Commons Attribution-NoDerivatives 4.0
% International License. To view a copy of this license, visit
% http://creativecommons.org/licenses/by-nd/4.0/ or send a letter to Creative
% Commons, PO Box 1866, Mountain View, CA 94042, USA.

%% Replace PAPERTITLE below with the paper title
\title{Paper Review Form (1000 words maximum)\\
    tr395: ZooKeeper: Wait-free coordination for Internet-scale systems \cite{ZooKeeper}}

\maketitle

\section*{Paper Summary}

\textsl{3--5 sentences. Briefly summarise the {\bf contributions} of the paper,
i.e.,~what it adds over the state of the art. Paraphrase and extract the
essentials rather than simply copying chunks of text. Be objective; later
sections allow for your own opinion.}

The main contribution of the paper is the creation of ZooKeeper, a service that
can be used to coordinate the processes of distributed applications. The main
difference to existing services, which either focus on coordinating specific
distributed applications or provide locking primitives which can be used to
create coordination services, is that ZooKeeper exposes an API that allows for
the development of such primitives using a wait-free model.

Finally, the paper demonstrates how ZooKeepers \textit{coordination kernel} can
be used to build higher level primitives in addition to providing real-world
use cases of ZooKeeper that have been successful at Yahoo!.

\section*{Pros and Cons}

\textsl{6 bullets. Succinctly state three positives and three negatives of the
paper.}

Pros:

\begin{itemize}

    \item The paper is the \textit{best of both worlds} in that it not only
    describes and evaluates a real-world system that has been effectively used
    at a large scale, the source-code to the ZooKeeper itself has been
    published and open to usage and investigation by others.

    \item Pro 2 TODO

    \item Pro 3 TODO

\end{itemize}

Cons:

\begin{itemize}

    \item A lot of the details of the evaluation are missing, for example, such
    as: How many trials were run for each experiment? What is the variance in
    throughput? Are the benchmarks themselves publicly available for
    reproducibility?

    \item Although the evaluation mentions how ZooKeeper compares to the
    published results of other, similar systems, having the direct comparisons
    in the paper would have been good as so many variables could differ between
    different papers.

    \item Con 3 TODO:

\end{itemize}

\section*{The Problem/Motivation}

\textsl{1--2 sentences per question. What is the motivation for the work, or
the problem being solved? Why is it important? If there is prior art, how was
it insufficient? If the problem had not previously been solved, why not?}

Coordinating distributed systems is a hard problem to solve, but one that is
constantly increasing importance due to the growth of internet services which
may even be distributed across the globe. The needs of different systems can
vary on case by case basis, and trade-offs always have to be made between using
off-the-shelf systems, stitching pre-built components together and building
custom solutions from scratch. Systems for the coordination of distributed
applications did exist, but due to their concerns and priorities the
performance ZooKeeper could use weak consistency guarantees in order to achieve
high performance.

\section*{The Solution/Approach}

\textsl{5--10 sentences. What have they done? How does it address the issues
set out above? How is it unique and/or innovative (if, indeed, it is)? Give
details, again using the paper as the source but again, not just copying text.
Instead, focus on paraphrasing/synopsising, and extracting the essential
details.}

TODO:


\section*{Evaluation}

\textsl{3--4 sentences. How do they evaluate their work? What questions does
their evaluation set out to answer? What does their evaluation say about the
strengths and weaknesses of their system? What is your opinion of the strengths
and weaknesses of the evaluation itself? Give highlights, not a point by point
reproduction of the evaluation section(s). In rare cases, systems papers may
not have any evaluation, in which case write `N/A' below.}

The authors evaluated the work by measuring the latency and throughput of
ZooKeeper in various configurations across a cluster of 50 servers - the number
of clients was always fixed at 35 whilst the up to 15 of the other servers were
used to provide the ZooKeeper service. The evaluation aims to answer the
question: \textit{How well does ZooKeeper perform}?

In my opinion the evaluation is the weakest part of the paper as it doesn't
really answer any questions about the system, although there are plenty of hard
numbers presented on the system, in isolation they tell nothing about how
ZooKeeper compares with its contemporaries. Although mention is made of how
ZooKeeper compares with a distributed-lock service called Chubby \cite{Chubby},
having read the paper referred to, the evaluation sections are non-comparable
in many ways, for example different hardware is used in both papers!

TODO: double-check above

\section*{Your Opinion}

\textsl{At least 3 sentences. This is the fun part where you get to judge both
the paper and the work it reports! Is the motivation convincing? The problem
important? The approach a good or bad idea? Why? Which specific things annoyed
you, or you thought were cool, or cool-but-flawed? Justify your opinions! Make
an argument which will convince others of your opinion.}

 In my opinion, the paper does a lot to convince the reader that ZooKeeper is
 an effective system through various use-cases and through the simple fact that
 it has been used with success at Yahoo!. Furthermore, ZooKeeper is now a
 top-level Apache Foundation project meaning it has seen further success beyond
 the life of the paper \cite{ApacheZooKeeper}.

 However, despite this, the paper really stumbles in its evaluation of
 ZooKeeper. TODO:

\section*{Questions for the Authors}

\textsl{Finally, imagine you're attending a talk about this paper given by one
of the authors. Give at least 2 questions that you would like to ask, specific
to the paper and the research it reports.}

What drove the decision to make ZooKeeper an open source Apache Foundation
project, has the project succeeded where you have wanted it to do so?

Do you think that the growth of data-centers has changed the nature the
distributed applications, or do you think the tools used need to rapidly change
and evolve to tackle an ever-developing problem?

\bibliographystyle{unsrt}

\bibliography{references}

\end{document}
