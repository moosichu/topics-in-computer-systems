\documentclass[11pt]{article}

\usepackage{a4wide, parskip}
\usepackage{cite}
\usepackage{hyperref}

\begin{document}

% This work is licensed under the Creative Commons Attribution-NoDerivatives 4.0
% International License. To view a copy of this license, visit
% http://creativecommons.org/licenses/by-nd/4.0/ or send a letter to Creative
% Commons, PO Box 1866, Mountain View, CA 94042, USA.

%% Replace PAPERTITLE below with the paper title
\title{Paper Review Form (1000 words maximum)\\
    tr395: In Search of an Understandable Consensus Algorithms \cite{Raft}}

\maketitle

\section*{Paper Summary}

\textsl{3--5 sentences. Briefly summarise the {\bf contributions} of the paper,
i.e.,~what it adds over the state of the art. Paraphrase and extract the
essentials rather than simply copying chunks of text. Be objective; later
sections allow for your own opinion.}

The largest contribution of the paper is the development of an alternative
consensus algorithm to Paxos \cite{Paxos}. However, the primary goal of the
algorithm is not to improve over Paxos from any kind of performance
perspective, but to provide an equivalent alternative baseline that is easier
for \textit{humans} to understand. The authors argue that it is difficult for
effective systems to be built by individuals who have little understanding of the
components involved, with Paxos being a common component of which people have
little understanding. Raft, is presented as the alternative, with the paper
providing an evaluation which aims to demonstrate its understandability.
Although other, similar alternative consensus algorithms exist, other
contributions are made in leadership, leader elections and membership changes.

\section*{Pros and Cons}

\textsl{6 bullets. Succinctly state three positives and three negatives of the
paper.}

Pros:

\begin{itemize}

    \item Having a focus on understandability is commendable, I have never come
    across systems papers with that before and I believe it is an important
    component of systems worth considering more often.

    \item The paper justifies its approach by taking examples from papers where
    the authors claim to use a protocol based on Paxos but with differences for
    \textit{real-world} systems \cite{Chubby}, demonstrating that in practise
    Paxos is rarely strictly adhered to.

    % \item The paper gives a summary of what it believes the properties of a
    % consensus algorithm should be, giving a good introduction to the subject
    % matter itself.

    \item The paper contains links to both LogCabin \cite{LogCabin}, the
    authors' implementation of the protocol, and to a web-page which gives an
    overview of the current state of the Raft protocol \cite{RaftIO}, including
    alternative implementations, lectures and publications on the protocol.

\end{itemize}

Cons:

\begin{itemize}

    \item Although acknowledging the existence of other non-Paxos consensus
    algorithms such as those found in ZooKeeper \cite{ZooKeeper} and
    Viewstamped Replication \cite{ViewstampedReplication}, their
    understandability isn't evaluated - only Raft and Paxos are directly
    compared in the user study the paper presents.

    \item The description of Raft, as presented in the paper, still has many
    subtleties due to potential unexpected interactions between components
    (despite the steps taken to decompose the protocol). Furthermore, as with
    the original Paxos paper, many implementation decisions are left to the
    reader \cite{RaftRefloated}.

    \item Some points are given without explanation. For example it is stated
    that a candidate will vote for itself without giving reasons \textit{why}
    this is necessary.

\end{itemize}

\section*{The Problem/Motivation}

\textsl{1--2 sentences per question. What is the motivation for the work, or
the problem being solved? Why is it important? If there is prior art, how was
it insufficient? If the problem had not previously been solved, why not?}

The principal argument is that Paxos is a hard protocol to understand - a
casual Google search \cite{PaxosSearch} on my behalf revealed that 7 out of the
top 10 results for \textit{Paxos Protocol} were devoted to
\textit{understanding Paxos} or \textit{Paxos made simple}, indicating that the
understandability of Paxos has been an issue.

This is arguably important because despite how effective a system may be in
theory, if those implementing the system have an incomplete understanding of
it, the probability of that system exhibiting its desirable qualities could
arguably decrease. The prior art itself has been so focused on explaining and
improving Paxos, that the main alternative consensus algorithms are part of
specific systems such as ZooKeeper \cite{ZooKeeper} or Viewstamped Replication
\cite{ViewstampedReplication}. This has resulted in a variety of unproven and
informal Paxos-based implementations being used in many real-world systems.

\section*{The Solution/Approach}

\textsl{5--10 sentences. What have they done? How does it address the issues
set out above? How is it unique and/or innovative (if, indeed, it is)? Give
details, again using the paper as the source but again, not just copying text.
Instead, focus on paraphrasing/synopsising, and extracting the essential
details.}

The paper provides a description of a consensus protocol that was created from
the ground-up as an alternative to Paxos. It is innovative in acknowledging
something that happens in practise - Paxos is rarely used in lieu of ad-hoc
alternatives, and formalising an alternative (Raft) which has been shown to be
correct. The paper also describes the process used in Raft's creation and what
drove various design decisions. The primary goal was to be able to decompose
the algorithms into various sub-problems that could be understood in isolation:
such as leadership elections, logging, and membership changes.

Furthermore, the paper analyses the performance of a concrete implementation of
Raft \cite{LogCabin} with the aim to demonstrate that the alternative is a
suitable replacement of Raft.

Finally, a usability study is done with the aim to demonstrate that Raft is
indeed more understandable than Paxos - participants learn about Raft and
Paxos, before completing quizzes on them and self-evaluating their
understanding of both.

\section*{Evaluation}

\textsl{3--4 sentences. How do they evaluate their work? What questions does
their evaluation set out to answer? What does their evaluation say about the
strengths and weaknesses of their system? What is your opinion of the strengths
and weaknesses of the evaluation itself? Give highlights, not a point by point
reproduction of the evaluation section(s). In rare cases, systems papers may
not have any evaluation, in which case write `N/A' below.}

The evaluation aims to answer three orthogonal questions about Raft: Is it
understandable? Is it performant? Is it correct?

With regards to correctness, links are provided for further reading on what has
been verified so far, and where work still needs to be completed (for example,
in proving the type safety of the specification).

With regards to performance, the main thing profiled is the time to detect and
replace crashed leaders - the paper simply states that Raft has similar
performance to other consensus algorithms. Therefore, the claims about
performance have to be taken at face-value, and that which is empirically
measured is specific to Raft, making it tough to compare the results against
other protocols.

With regards to understandability, a study was performed using 43 participants
where they were each taught about both Raft and Paxos by watching a video prior
to self-evaluating their understanding of each and completing a quiz. Overall,
the participants performed better in the quiz on Raft and felt it was easier to
understand than Paxos.

The paper claims that the experiments favoured Paxos in two ways: some
participants reported having prior experience with Paxos and the Paxos video
was 14\% longer than the Raft video. However, no evidence is given as to why
these points favour Paxos over Raft. For example, some evidence shows that
people with prior exposure to topics are more likely to have misconceptions
about them which may negatively affect test scores after being taught material
\cite{DerekMuller2008}.

% The paper also states that participants may have been biased by knowledge of
% the hypothesis that Raft is easier to understand when performing their
% self-evaluation, maybe steps should have been taken to ensure candidates were
% not aware of the hypothesis being tested.

Finally, a single lecturer is used for for both videos, which arguably doesn't
eliminate many possible biases - ideally a random sample of lecturers should
have been used to teach each algorithm in their own style. Although there are
practical issues in doing so, I feel the paper should have stated that having a
single lecturer was far from ideal.


\section*{Your Opinion}

\textsl{At least 3 sentences. This is the fun part where you get to judge both
the paper and the work it reports! Is the motivation convincing? The problem
important? The approach a good or bad idea? Why? Which specific things annoyed
you, or you thought were cool, or cool-but-flawed? Justify your opinions! Make
an argument which will convince others of your opinion.}

Considering how subjective \textit{understandability} is, there are naturally
going to be varying opinions from the perspective of the reader on how
effective Raft is at providing an understandable alternative to the Paxos
consensus protocol. In my opinion, I found the paper itself, and the
description of Raft, a very pleasant read and incredibly understandable - so
from my perspective I feel that the authors succeeded in achieving their aim of
making something understandable. Furthermore, I also agree with all of the
points raised in the paper on the importance of understandability, especially
as real-world uses of Paxos can deviate from the original proposal
significantly \cite{Chubby} \cite{ZooKeeper}. Therefore many implementations of
Paxos haven't been mathematically proven to be safe consensus protocols!

However, in trying to demonstrate how understandable the Raft consensus
protocol is, the paper falls down in its evaluation and usability study for the
reasons mentioned above. The brevity of the evaluation section as a whole is
surprising, considering that three distinct criteria are evaluated. Therefore,
it is the weakest part of the paper. This opinion is further justified by the
fact that attempts to recreate some of the results presented were proven more
difficult than anticipated \cite{RaftRefloated}.

\section*{Questions for the Authors}

\textsl{Finally, imagine you're attending a talk about this paper given by one
of the authors. Give at least 2 questions that you would like to ask, specific
to the paper and the research it reports.}

Now the Raft has been released to the wider world and had time to mature, do
you feel like its existence has had the positive impact you hoped it would?

In hindsight, do you feel that there are aspects of Raft that you would change
to make them more understandable compared to what you ultimately settled on?

Are there any other protocols/algorithms you feel are due an understandability
update?

\bibliographystyle{unsrt}

\bibliography{references}

\end{document}
