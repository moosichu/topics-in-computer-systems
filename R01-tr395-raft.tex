\documentclass[11pt]{article}

\usepackage{a4wide, parskip}
\usepackage{cite}
\usepackage{hyperref}

\begin{document}

% This work is licensed under the Creative Commons Attribution-NoDerivatives 4.0
% International License. To view a copy of this license, visit
% http://creativecommons.org/licenses/by-nd/4.0/ or send a letter to Creative
% Commons, PO Box 1866, Mountain View, CA 94042, USA.

%% Replace PAPERTITLE below with the paper title
\title{Paper Review Form (1000 words maximum)\\
    tr395: In Search of an Understandable Consensus Algorithms \cite{Raft}}

\maketitle

\section*{Paper Summary}

\textsl{3--5 sentences. Briefly summarise the {\bf contributions} of the paper,
i.e.,~what it adds over the state of the art. Paraphrase and extract the
essentials rather than simply copying chunks of text. Be objective; later
sections allow for your own opinion.}

The largest contribution of the paper is the development of an alternative
consensus algorithm to Paxos \cite{Paxos}. However, the primary goal of the
algorithm isn't to improve over Paxos from any kind of performance perspective,
but to provide and equivalent alternative that is easier for \textit{humans} to
understand. The authors argue that it is difficult for effective systems by
built by individuals who have little understanding of the components involved,
with Paxos being a common component of which people have little understanding.
Raft, is presented as the alternative, with the paper providing an evaluation
which aims to demonstrate its understandability. Although other, similar
alternative consensus algorithms exist, other contributions are made in
leadership, leader elections and membership changes. TODO, check word is
spelled correctly)

\section*{Pros and Cons}

\textsl{6 bullets. Succinctly state three positives and three negatives of the
paper.}

Pros:

\begin{itemize}

    \item Having a focus on understandability is commendable, I have never come
    across systems papers with that before and I believe it is an important
    component of systems worth considering more often.

    \item Pro 2

    \item Pro 3

\end{itemize}

Cons:

\begin{itemize}

    \item Although acknowledging the existence of other non-Paxos consensus
    algorithms such as those found in ZooKeeper \cite{ZooKeeper} and VR, their
    understandability isn't evaluated - only Raft and Paxos are directly
    compared.

    \item Con 2

    \item Con 3

\end{itemize}

\section*{The Problem/Motivation}

\textsl{1--2 sentences per question. What is the motivation for the work, or
the problem being solved? Why is it important? If there is prior art, how was
it insufficient? If the problem had not previously been solved, why not?}

The main argument made is that Paxos is a hard protocol to understand - a
casual Google search \cite{PaxosSearch} on my behalf revealed that 7 out of the
top 10 results for \textit{Paxos Protocol} were devoted to
\textit{understanding Paxos} or \textit{Paxos made simple}, indicating that the
understandability of Paxos has been an issue.

This is arguably important because despite how effective a system may be in
theory, if those implementing the system have an incomplete understanding it,
the probability of that system exhibiting its desirable qualities could
arguably decrease. The prior art itself has been so focused on explaining and
improving Paxos, that the main alternative consensus algorithms are part of
specific systems such as ZooKeeper or VR.

\section*{The Solution/Approach}

\textsl{5--10 sentences. What have they done? How does it address the issues
set out above? How is it unique and/or innovative (if, indeed, it is)? Give
details, again using the paper as the source but again, not just copying text.
Instead, focus on paraphrasing/synopsising, and extracting the essential
details.}


\section*{Evaluation}

\textsl{3--4 sentences. How do they evaluate their work? What questions does
their evaluation set out to answer? What does their evaluation say about the
strengths and weaknesses of their system? What is your opinion of the strengths
and weaknesses of the evaluation itself? Give highlights, not a point by point
reproduction of the evaluation section(s). In rare cases, systems papers may
not have any evaluation, in which case write `N/A' below.}


\section*{Your Opinion}

\textsl{At least 3 sentences. This is the fun part where you get to judge both
the paper and the work it reports! Is the motivation convincing? The problem
important? The approach a good or bad idea? Why? Which specific things annoyed
you, or you thought were cool, or cool-but-flawed? Justify your opinions! Make
an argument which will convince others of your opinion.}
below


\section*{Questions for the Authors}

\textsl{Finally, imagine you're attending a talk about this paper given by one
of the authors. Give at least 2 questions that you would like to ask, specific
to the paper and the research it reports.}

\bibliographystyle{unsrt}

\bibliography{references}

\end{document}
