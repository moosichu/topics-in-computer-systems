\documentclass[11pt]{article}

\usepackage{a4wide, parskip}
\usepackage{cite}
\usepackage{hyperref}

\begin{document}

% This work is licensed under the Creative Commons Attribution-NoDerivatives 4.0
% International License. To view a copy of this license, visit
% http://creativecommons.org/licenses/by-nd/4.0/ or send a letter to Creative
% Commons, PO Box 1866, Mountain View, CA 94042, USA.

%% Replace PAPERTITLE below with the paper title
\title{Paper Review Form (1000 words maximum)\\
    tr395: Fast Byte-Granularity Software Fault Isolation \cite{BGI}}

\maketitle

\section*{Paper Summary}

\textsl{3--5 sentences. Briefly summarise the {\bf contributions} of the paper,
i.e.,~what it adds over the state of the art. Paraphrase and extract the
essentials rather than simply copying chunks of text. Be objective; later
sections allow for your own opinion.}

BGI (Byte-Granularity Isolation) is introduced, a fault-isolation technique
that aims to isolate third-party kernel extensions (such as drivers) from the
kernel of an operating system itself. The reason for this is that these
extensions are one of the main causes for poor reliability in operating systems
\cite{OSErrors}. This system aims to add over the state of the art by having
relatively strong isolation guarantees in addition to low overhead, and
therefore more amenable to practical usage. Furthermore, BGI works entirely in
software, therefore not requiring custom hardware in order to operate as
desired. As the name implies, fault isolation is handled at a per-byte level.


\section*{Pros and Cons}

\textsl{6 bullets. Succinctly state three positives and three negatives of the
paper.}

Pros:

\begin{itemize}

    \item BGI appears to make strides in solving a very real problem - the blue
    screen of death.

    \item The paper is honest about what BGI does \textit{not} protect against,
    such as deliberately malicious code.

    \item Not requiring specialised hardware is good for helping this system
    see uptake.

\end{itemize}

Cons:

\begin{itemize}

    \item Although isolating one extension this way may cause an overall
    decrease in performance of single-digit percentages, what would the effect
    be if \textit{all} extensions were isolated? This could be much worse a
    expected.

    \item Although it doesn't require any changes to the source code of an
    extension, recompilation does mean that ultimately it is up to an extension
    provider to ensure they compile their code with this compiler. Therefore
    older and unsupported extensions are unlikely to be able to achieve desired
    protections under this system.

    \item Due to the per-byte level granularity of BGI, a 12.5\% memory
    overhead is a rather significant one, also opening up the potential for
    pathological cache misses in some cases which could be important for
    performance-sensitive extensions.


\end{itemize}

\section*{The Problem/Motivation}

\textsl{1--2 sentences per question. What is the motivation for the work, or
the problem being solved? Why is it important? If there is prior art, how was
it insufficient? If the problem had not previously been solved, why not?}

The motivation for the work is that kernel extensions are often buggy and the
cause of operating system failures \cite{OSErrors}. Therefore, isolating
extensions such that these errors don't drag down an entire system with them is
an important problem to solve. Although there have been previous approaches to
tackling this problem, the authors argue that they suffer from various issues
that make them unsuitable for general usage. These include large overheads,
specialised hardware requirements, weak isolation guarantees and/or the need to
manually modify the extensions to support the isolation systems.

\section*{The Solution/Approach}

\textsl{5--10 sentences. What have they done? How does it address the issues
set out above? How is it unique and/or innovative (if, indeed, it is)? Give
details, again using the paper as the source but again, not just copying text.
Instead, focus on paraphrasing/synopsising, and extracting the essential
details.}

The idea behind BGI is that kernel extension writers can use a customised
compiler on their extensions to get the isolation protection BGI aims to
provide. BGI works by associating an access control list of \textit{read},
\textit{write}, \textit{icall}, \textit{type}, and \textit{ownership} with
every byte of virtual memory that kernel extensions can interact with. Kernel
calls are wrapped using an \textit{interposition library} to ensure values
returned from the kernel have appropriate permissions set, and to ensure
arguments passed to the kernel satisfy necessary permission conditions. The
compiler inserts code in support of managing access control as well. BGI runs
code compiled this way in one of several \textit{untrusted domains} - for
logistical reasons, some inter-dependent extensions may need to share the same
domain. Finally, a dynamic type system is a crucial part of the system to
ensure permissions are correctly adhered to.

One point of note is that the system does not protect against malicious code,
it simply helps prevent errors in extensions from bringing down an entire
operating system.

\section*{Evaluation}

\textsl{3--4 sentences. How do they evaluate their work? What questions does
their evaluation set out to answer? What does their evaluation say about the
strengths and weaknesses of their system? What is your opinion of the strengths
and weaknesses of the evaluation itself? Give highlights, not a point by point
reproduction of the evaluation section(s). In rare cases, systems papers may
not have any evaluation, in which case write `N/A' below.}

The evaluation consists of three parts aiming to answer: How effective is BGI and
what performance overhead does it incur?

The first section consists of intentionally inserting bugs into two device
drivers and seeing if BGI successfully contains the resulting \textit{blue
screen failures} or hangs. A further experiment tests how successfully BGI can
then recover from certain failures as well. This section is fairly successful
at conveying how BGI can help contain certain errors - however, due to the
seemingly automated nature of bug-insertion and testing, the fact that only two
drivers were tested for the injected-fault section seems rather small.

The next section is the performance evaluation. Although all performance
results are the average of at least 3 runs, unfortunately no statistical
information is presented. Although the overhead introduced by BGI seems
relatively small for the range of extensions tested (mostly single-digit
percentages), what is unanswered is how those overheads could stack if multiple
extensions have BGI enabled.

The final section analyses BGI's ability to find \textit{real bugs} by using
its testing capabilities. The results simply list the number of bugs found and
their nature across an unknown number of extensions. It gets the point across
but lacks enough information to comment on any further.


\section*{Your Opinion}

\textsl{At least 3 sentences. This is the fun part where you get to judge both
the paper and the work it reports! Is the motivation convincing? The problem
important? The approach a good or bad idea? Why? Which specific things annoyed
you, or you thought were cool, or cool-but-flawed? Justify your opinions! Make
an argument which will convince others of your opinion.}

Overall, I think the paper is clear and the work it reports is valuable,
especially having had personal experience with the problem at-hand (the blue
screen of death).

The compile-time nature of the system is probably the most problematic part of
it. As the system isn't a simple update to a kernel, there is a dependency on
extension writers to re-compile their code and distribute the updates.
Furthermore, due to the static nature of BGI, if there are bugs in the compiler
itself, distributing bug-fixes and updates due to errors caused by them is a
hard problem to solve.

Although the paper goes really in-depth into the mechanisms of the system, this
is probably necessary due to the closed-source nature of the system.


\section*{Questions for the Authors}

\textsl{Finally, imagine you're attending a talk about this paper given by one
of the authors. Give at least 2 questions that you would like to ask, specific
to the paper and the research it reports.}

Since the publication of the paper, has BGI seen use in Windows, and if not,
why not?

Would using a binary rewriter installed into the kernel that can update
extensions itself be a good idea?

\bibliographystyle{unsrt}

\bibliography{references}

\end{document}
