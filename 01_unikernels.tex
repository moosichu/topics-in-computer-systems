\documentclass[11pt]{article}

\usepackage{a4wide, parskip}
\usepackage{cite}
\usepackage{hyperref}

\begin{document}

%% Replace PAPERTITLE below with the paper title
\title{Paper Review Form (1000 words maximum)\\
    tr395: Unikernels: Library Operating Systems for the Cloud \cite{Unikernels}}

\maketitle

\section*{Paper Summary}

\textsl{3--5 sentences. Briefly summarise the {\bf contributions} of the paper,
i.e.,~what it adds over the state of the art. Paraphrase and extract the
essentials rather than simply copying chunks of text. Be objective; later
sections allow for your own opinion.}

TODO


\section*{Pros and Cons}

\textsl{6 bullets. Succinctly state three positives and three negatives of the
paper.}

Pros:

\begin{itemize}

    \item XeN is

\end{itemize}

Cons:

\begin{itemize}

    \item XeN is falling out of favour (TODO: reference).

    %    https://www.theregister.co.uk/2017/11/07/aws_writes_new_kvm_based_hypervisor_to_make_its_cloud_go_faster/


    \item The world has moved on from VMs to containers

    \item KVM is lightweight? TODO: check KVM support for this!

    \item OCaml, despite its features, isn't a pragmatic language choice

\end{itemize}


\section*{The Problem/Motivation}

\textsl{1--2 sentences per question. What is the motivation for the work, or
the problem being solved? Why is it important? If there is prior art, how was
it insufficient? If the problem had not previously been solved, why not?}

The way internet services have been used and deployed has evolved and developed
significantly over the past few years, mainly with the growing use of
hypervisor and many heavyweight Virtual Machines. However, spawning complete
instances operating systems (such as GNU/Linux, or Windows) which then execute
a single task (eg. a web server, or a database) could be considered wasteful.

The problem is important because of the scale at which these systems operate,
with estimates that Amazon's EC2 is made up almost 500,000 servers
\cite{EC2Amount}. Any kind of resource savings, in computation, power or
memory, however minor for individual instances, can have major consequences
at scale.

TODO: PRIOR art

The thing about the problem in question, of resources being question, is a
problem where incremental improvements could always be possible and beneficial.
Due to the rapidly advancing nature of the systems in question, optimisations
will be possible for the foreseeable future. There is also the question of
tradeoffs, where the number of variables in play mean that some solutions will
always have some advantages compared to others based on the tradeoffs they
make.


\section*{The Solution/Approach}

\textsl{5--10 sentences. What have they done? How does it address the issues
set out above? How is it unique and/or innovative (if, indeed, it is)? Give
details, again using the paper as the source but again, not just copying text.
Instead, focus on paraphrasing/synopsising, and extracting the essential
details.} %% Type your text below

TODO:


\section*{Evaluation}

\textsl{3--4 sentences. How do they evaluate their work? What questions does
their evaluation set out to answer? What does their evaluation say about the
strengths and weaknesses of their system? What is your opinion of the strengths
and weaknesses of the evaluation itself? Give highlights, not a point by point
reproduction of the evaluation section(s). In rare cases, systems papers may
not have any evaluation, in which case write `N/A' below.}

TODO:


\section*{Your Opinion}

\textsl{At least 3 sentences. This is the fun part where you get to judge both
the paper and the work it reports! Is the motivation convincing? The problem
important? The approach a good or bad idea? Why? Which specific things annoyed
you, or you thought were cool, or cool-but-flawed? Justify your opinions! Make
an argument which will convince others of your opinion.}

TODO:


\section*{Questions for the Authors}

\textsl{Finally, imagine you're attending a talk about this paper given by one
of the authors. Give at least 2 questions that you would like to ask, specific
to the paper and the research it reports.}

If taking a clean-slate approach to this project again, what would you change
based on what you learnt while making it?

\bibliographystyle{unsrt}

\bibliography{references}

\end{document}
