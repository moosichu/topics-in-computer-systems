\documentclass[11pt]{article}

\usepackage{a4wide, parskip}
\usepackage{cite}
\usepackage{hyperref}

\begin{document}

% This work is licensed under the Creative Commons Attribution-NoDerivatives 4.0
% International License. To view a copy of this license, visit
% http://creativecommons.org/licenses/by-nd/4.0/ or send a letter to Creative
% Commons, PO Box 1866, Mountain View, CA 94042, USA.

%% Replace PAPERTITLE below with the paper title
\title{Paper Review Form (1000 words maximum)\\
    tr395: Container-based Operating System Virtualization: A Scalable, High-performance Alternative to Hypervisors \cite{Containers}}

\maketitle

\section*{Paper Summary}

\textsl{3--5 sentences. Briefly summarise the {\bf contributions} of the paper,
i.e.,~what it adds over the state of the art. Paraphrase and extract the
essentials rather than simply copying chunks of text. Be objective; later
sections allow for your own opinion.}

The biggest contribution of the paper is the formalisation, description and
presentation of container-based operating systems for the academic community.
It takes the form of an analysis of a container-based operating system known as
Linux-VServer which has been in development since 2008 \cite{LinuxVServer}. The
paper aims to present a balanced view of the pros and cons of container-based
systems compared with alternatives such as hypervisors, whilst providing a
technical breakdown and evaluation of VServer itself.

\section*{Pros and Cons}

\textsl{6 bullets. Succinctly state three positives and three negatives of the
paper.}

Pros:

\begin{itemize}

    \item Comprehensive analyses of existing systems are valuable research in
    and of themselves - providing that for container-based operating systems
    gives this paper a lot of value.

    \item The paper pragmatically weighs up the pros and cons of
    container-based operating systems versus hypervisors, the goal is clearly
    to provide an overview and explanation of them instead of advocating for or
    against them.

    \item The paper is understandable and self-contained, not much reference
    chasing is required.

\end{itemize}

Cons:

\begin{itemize}

    \item The title is slightly misleading, although parts of the paper are
    about container-based operating systems in general, the focus is very much
    on one specific implementation: Linux-VServer.

    \item Some terms are introduced and never explained or referenced. A good
    example of this is \textit{the Grid}, which is repeatedly mentioned in the
    paper and something I have been unable to source.

    \item In benchmarking the performance of hypervisors and containers, only
    the Xen hypervisor and Linux-VServer are compared, which considering the
    other options mentioned in the paper (VMware ESX, Virtual Server,
    Solaris 10, and Virtuozzo), is an odd choice and gives the reader a narrow
    view of the state of the art.

\end{itemize}

\section*{The Problem/Motivation}

\textsl{1--2 sentences per question. What is the motivation for the work, or
the problem being solved? Why is it important? If there is prior art, how was
it insufficient? If the problem had not previously been solved, why not?}

The motivation the container-based operating systems is to provide a
lightweight alternative to virtual machines running on a hypervisor. Although
there are benefits to this, the paper argues that in some contexts having many
kernel instances (especially of identical kernels) is wasteful, and a container
based OS is a way to have a kernel shared by many different pseudo-VMs.
Arguments are also made that other overheads related to hypervisors can be
reduced or eliminated by using containers as well.

The work is important because of the explosion of VMs and hypervisors in data
centers \cite{VirtualizationGrowth}, and whilst the dramatic growth of them has
been driven by many benefits they provide, finding ways to reap similar
benefits whilst reducing some of the costs could be extremely valuable.

The prior art is what the paper describes, the systems being developed to solve
the problem the paper analyses are in constant development - incremental and
major gains are constantly being made in what is both an incredibly active area
of research and engineering \cite{GoogleScholarVirtualization}. There are many
different solutions to the problems, each involving different trade-offs, a
contributing factor to why solving the problems described is so hard, is
because the nature of the problem is constantly changing. Hardware, software,
services, scale, what is expected, use cases, alternatives, and more are
constantly under flux, of which the systems being described a singular
component.

\section*{The Solution/Approach}

\textsl{5--10 sentences. What have they done? How does it address the issues
set out above? How is it unique and/or innovative (if, indeed, it is)? Give
details, again using the paper as the source but again, not just copying text.
Instead, focus on paraphrasing/synopsising, and extracting the essential
details.}

The paper describes the main benefits of virtualization being those of
isolation. For example, many virtual machines running over a single hypervisor
are \textit{isolated} from each other in various ways. Whilst there are other
potential benefits to virtualization, which container-based operating systems
may not be able to provide, the paper focuses on how it is possible to provide
isolation between containers running on a singular kernel with the caveat being
that some of the stronger isolation guarantees hypervisors provide are exchanged
for higher efficiency.

The three types of isolation described are those of \textit{fault},
\textit{resource} and \textit{security} isolation. Linux-VServer, which the
paper describes, provides these through modifications to the Linux kernel which
allow for \textit{Fair Share} scheduling of various resources such as the CPU
and I/O. Interesting techniques are also used such as resetting the Virtual
Machine that uses the most memory when the swap has almost filled. In other
cases such as networking, limitations are introduced such as the inability for
VMs to change their own route table entries or IP tables rules - in exchange
for native Linux networking performance.

\section*{Evaluation}

\textsl{3--4 sentences. How do they evaluate their work? What questions does
their evaluation set out to answer? What does their evaluation say about the
strengths and weaknesses of their system? What is your opinion of the strengths
and weaknesses of the evaluation itself? Give highlights, not a point by point
reproduction of the evaluation section(s). In rare cases, systems papers may
not have any evaluation, in which case write `N/A' below.}

The evaluation mostly consists of quantative comparisons of Linux-VServer and
the Xen hypervisor using various benchmarks. Three styles of benchmarks are
presented: micro, system and macro. The goal of the evaluation is to
demonstrate the advantages container-based operating systems have over
hypervisors with regards to performance.

For the most part, the evaluation succeeds in showing this for many aspects of
container-based operating systems, including system call timings, network
bandwidth, disk performance, CPU performance, and realistic loads under the
Open Source Database Benchmark \cite{OSDB}.

The evaluation does have a few weaknesses. One benchmark aims to demonstrate
the difference in performance between Linux-VServer and the Xen hypervisor when
8 VMs are run simultaneously, with one of the VMs being given a reservation of
1/4th of the overall CPU time. However, Xen doesn't support the ability to
explicitly do this, meaning no useful direct comparison could be made. Whilst
having the feature could be considered an advantage of Linux-VServer,
benchmarking it doesn't provide much insight as the results obtained are not
comparable.

However, the biggest weakness of the evaluation is how focused it is on
Linux-VServer. Despite the paper mentioning a variety of hypervisors and
container-based operating systems, including VMware ESX, Virtual Server,
Solaris 10, and Virtuozzo, only Linux-VServer and Xen hypervisor are compared,
making the evaluations themselves non-comprehensive.

\section*{Your Opinion}

\textsl{At least 3 sentences. This is the fun part where you get to judge both
the paper and the work it reports! Is the motivation convincing? The problem
important? The approach a good or bad idea? Why? Which specific things annoyed
you, or you thought were cool, or cool-but-flawed? Justify your opinions! Make
an argument which will convince others of your opinion.}

Considering the context in which the paper was published, I think the paper
provides value in providing an overview of container-based operating system to
those who are unfamiliar with them. It presents convincing arguments for and
against their usage versus hypervisors, with a focus on isolation and
efficiency. The main aspects of the paper which in my opinion fall short, are
the evaluations themselves, as described above. One of the things the paper
could have done would have been exploring and evaluating the case of containers
inside of a virtual machine - containers and virtual machines do have some
orthogonal benefits which complement each other quite well, so observing how a
multiple containers scale on a virtual machine running on a hypervisor could
have formed part of an interesting evaluation.

Finally, in hindsight the paper's focus on Linux-VServer is curious - as
despite the recent explosion in container-based systems such as Docker and LXC,
Linux-VServer doesn't seem to have enjoyed as much popularity
\cite{ContainerGrowth}.

\section*{Questions for the Authors}

\textsl{Finally, imagine you're attending a talk about this paper given by one
of the authors. Give at least 2 questions that you would like to ask, specific
to the paper and the research it reports.}

What do you think about the current state of containers and the different
options available?

What are the biggest challenges facing containers, and do you see any
convergence happening with virtualization technologies in the future?

\bibliographystyle{unsrt}

\bibliography{references}

\end{document}
