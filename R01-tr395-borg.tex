\documentclass[11pt]{article}

\usepackage{a4wide, parskip}
\usepackage{cite}
\usepackage{hyperref}

\begin{document}

% This work is licensed under the Creative Commons Attribution-NoDerivatives 4.0
% International License. To view a copy of this license, visit
% http://creativecommons.org/licenses/by-nd/4.0/ or send a letter to Creative
% Commons, PO Box 1866, Mountain View, CA 94042, USA.

%% Replace PAPERTITLE below with the paper title
\title{Paper Review Form (1000 words maximum)\\
    tr395: Large-scale cluster management at Google with Borg \cite{Borg}}

\maketitle

\section*{Paper Summary}

\textsl{3--5 sentences. Briefly summarise the {\bf contributions} of the paper,
i.e.,~what it adds over the state of the art. Paraphrase and extract the
essentials rather than simply copying chunks of text. Be objective; later
sections allow for your own opinion.}

The main contribution of the paper is the description and analysis of a
proprietary cluster management system at Google called Borg. The paper
describes how Borg works from a user perspective and architecturally, before
discussing availability and evaluating utilisation. Finally, the isolation
systems are described before lessons learned from the system are listed.

\section*{Pros and Cons}

\textsl{6 bullets. Succinctly state three positives and three negatives of the
paper.}

Pros:

\begin{itemize}

    \item The paper gives an in-depth overview of a feat of engineering that has
    been used successfully at Google for over a decade.

    \item Although Borg is proprietary to Google, the paper describing it
    provides a useful insight into \textit{what goes on behind the curtains},
    that would otherwise be unavailable.

    \item The lessons learned section help quantify what architects of other
    systems can takeaway from the experience Google had with Borg.

\end{itemize}

Cons:

\begin{itemize}

    \item Borg is a proprietary system, so although the paper may be
    informative, it's not describing something that anyone will use unless they
    work at Google.

    \item The evaluation methodology seems to depend an a familiarity with Borg
    features such as Fauxmaster, and therefore it is unclear what was simulated
    and what wasn't.

    \item Applications running on Borg are expected to handle many types of
    failures themselves, with no features provided to handle things such as
    storing persistent state.

    \item Latency-sensitive applications seem to get large priorities over
    batch-classed applications with no drawbacks. The paper glosses over what
    is done to prevent all application developers from marking their
    applications as latency-sensitive despite mentioning other tricks internal
    developers did pull to try and give themselves access to the most
    resources.

\end{itemize}

\section*{The Problem/Motivation}

\textsl{1--2 sentences per question. What is the motivation for the work, or
the problem being solved? Why is it important? If there is prior art, how was
it insufficient? If the problem had not previously been solved, why not?}

The motivation for the work is that Google, as a company, owns and runs many
servers which are used to provide many services, and wants to ensure
application developers don't have to worry about how their applications are
going to effectively utilise the hardware. The work important internally at
Google, and is relevant to the wider community as the open-source platform
Kubernetes, is a result of the lessons learned from Borg. Although there is
prior in providing many of the features Borg does, Borg is a unique in
operating successfully at such a large scale for a relatively long time (10
years).

\section*{The Solution/Approach}

\textsl{5--10 sentences. What have they done? How does it address the issues
set out above? How is it unique and/or innovative (if, indeed, it is)? Give
details, again using the paper as the source but again, not just copying text.
Instead, focus on paraphrasing/synopsising, and extracting the essential
details.}

Google developed a cluster management system, which developed and evolved over
10 years, developing a wide feature set making it suitable for being used at an
industrial scale. Borg consists of many independent \textit{cells}, with the
median cell size being around 10K (heterogeneous) machines. A cell is managed
by a controller called the Borgmaster, and each machines has a Borglet process
that communicates with the Borgmaster. Users submit jobs to cells, with each
job consisting of one or more tasks, identical binary programs.

The thing that makes Borg unique is simply the scale at which it operates at,
both in feature-set, time and number of services it supports. Interestingly,
nothing was actively planned to ensure it would endure like that, it was simply
the case that when bottlenecks were being reached, as-need improvements to Borg
ensured that scale never became an issue.

\section*{Evaluation}

\textsl{3--4 sentences. How do they evaluate their work? What questions does
their evaluation set out to answer? What does their evaluation say about the
strengths and weaknesses of their system? What is your opinion of the strengths
and weaknesses of the evaluation itself? Give highlights, not a point by point
reproduction of the evaluation section(s). In rare cases, systems papers may
not have any evaluation, in which case write `N/A' below.}

The paper focuses on evaluating how effectively Borg utilises Google's many
machines. However, the evaluation isn't a comparison of Borg against other,
similar systems, it is an evaluation and comparison of how effective different
choices of policies compare against the ones Borg uses (essentially, Borg is
compared with non-optimal configurations of Borg in order to demonstrate that
those configurations are non-optimal). The evaluation is strong in that is does
contain a lot of data, but I feel it is weak in not containing much useful
information. It is oddly structured, and I still feel like I don't understand
many parts of it. Furthermore, claims are made about many of the non-optimal
set-ups without justification, in order to demonstrate why Borg is better then
other systems (for example, it is stated that many other organisations run
batch and user-facing jobs on separate machines, but no evidence is given for
this).

\section*{Your Opinion}

\textsl{At least 3 sentences. This is the fun part where you get to judge both
the paper and the work it reports! Is the motivation convincing? The problem
important? The approach a good or bad idea? Why? Which specific things annoyed
you, or you thought were cool, or cool-but-flawed? Justify your opinions! Make
an argument which will convince others of your opinion.}

From a historical perspective, I believe the paper does a good job of giving an
overview of an impressive feat of engineering which has powered the systems of
one of the largest companies on the internet. However, my takeaway is that is
much more an overview and documentation of something that came to exist, as
opposed to the paper presenting a piece of concrete research from which can
be learnt.

It's a difficult paper to comment on because the of the scale of what is being
described, Borg is essentially the composition of many systems into a singular
whole with 10 years of legacy.

Although the paper is informative, I did find the evaluation section confusing
as it wasn't clear what the overall takeaway was meant to be. The lack of
conclusion to the paper itself mad it a slightly tougher read.

\section*{Questions for the Authors}

\textsl{Finally, imagine you're attending a talk about this paper given by one
of the authors. Give at least 2 questions that you would like to ask, specific
to the paper and the research it reports.}

Is the plan for Kubernetes to replace Borg, or live alongside it?

Why was the decision made to create Kubernetes instead of open-sourcing Borg?

\bibliographystyle{unsrt}

\bibliography{references}

\end{document}
