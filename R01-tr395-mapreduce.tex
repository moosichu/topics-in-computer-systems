\documentclass[11pt]{article}

\usepackage{a4wide, parskip}
\usepackage{cite}
\usepackage{hyperref}

\begin{document}

% This work is licensed under the Creative Commons Attribution-NoDerivatives 4.0
% International License. To view a copy of this license, visit
% http://creativecommons.org/licenses/by-nd/4.0/ or send a letter to Creative
% Commons, PO Box 1866, Mountain View, CA 94042, USA.

%% Replace PAPERTITLE below with the paper title
\title{Paper Review Form (1000 words maximum)\\
    tr395: MapReduce: Simplified Data Processing on Large Clusters \cite{MapReduce}}

\maketitle

\section*{Paper Summary}

\textsl{3--5 sentences. Briefly summarise the {\bf contributions} of the paper,
i.e.,~what it adds over the state of the art. Paraphrase and extract the
essentials rather than simply copying chunks of text. Be objective; later
sections allow for your own opinion.}

The paper introduces a technique called MapReduce for processing large amounts
of data across a distributed cluster of computers. The technique works by
providing an abstraction that allows users of the MapReduce system to implement
two functions: a \textit{map} function and a \textit{reduce} function. Once
implemented, the system will automatically handle running these functions
across a large number of failure-prone computers whilst handling the
complexities associated with distributed computing. By requiring that functions
the user implements adhere to a certain format, MapReduce aims to simplify
computations on data at this scale. The paper also describes and evaluates an
implementation of MapReduce created by Google for their internal usage.


\section*{Pros and Cons}

\textsl{6 bullets. Succinctly state three positives and three negatives of the
paper.}

Pros:

\begin{itemize}

    \item The idea is simple to understand, yet powerful and useful.

    \item How some of the most obvious failure cases are dealt with is
    explained in a straight-forward manner.

    \item The discussion of edge-cases that appeared using MapReduce in order
    to allow future implementers of similar systems to anticipate them is
    appreciated.

\end{itemize}

Cons:

\begin{itemize}

    \item The master is a single point of failure, even though the case of the
    master failing is hand-waved away - there isn't much done to address its
    potential as a bottleneck in the face of functional-but-slow behaviour.

    \item Using C++ as the language for Map and Reduce functions may not be the
    best choice as it allows programmers to easily break out of the restricted
    programming model and break the assumptions of the MapReduce system.
    However, the potential benefits of an alternative, such as a
    domain-specific language with desired guarantees are never mentioned.

    \item The behaviour of the distributed file system used to help provide
    many of the properties of MapReduce isn't really discussed despite it being
    so crucial the correct and efficient operation of MapReduce.

\end{itemize}

\section*{The Problem/Motivation}

\textsl{1--2 sentences per question. What is the motivation for the work, or
the problem being solved? Why is it important? If there is prior art, how was
it insufficient? If the problem had not previously been solved, why not?}

The motivation for the work is the fact that internally at Google, programmers
were required to expend a lot of time programming special-purpose systems for
running their computations across the large clusters of machines belonging to
Google. However, the computations themselves were conceptually simple, so this
approach seemed like wasted effort - so an attempt was made to generalise the
distribution of computations. The problem is important because of the growth of
distributed systems, which has only exploded with the increase in scope, scale
and number of datacenters since the paper was published
\cite{VirtualizationGrowth}.

The reason the problem hadn't been previously solved is because although
distributed systems and networking had been around for a while, the pressing
need to actually perform distributed computations at this scale on cheap,
commodity hardware in a controlled environment was a relatively novel one.
Furthermore, although other distributed systems did exist - they either
provided primitives and other functionality to allow users to more easily
implement their own fault-tolerance behaviour in custom programs (therefore
being hard to use) \cite{River} \cite{MPI}
\cite{BridgingModelForParallelComputation}, or they were simply an
implementation of specific algorithm such as a sort \cite{NowSort}.

\section*{The Solution/Approach}

\textsl{5--10 sentences. What have they done? How does it address the issues
set out above? How is it unique and/or innovative (if, indeed, it is)? Give
details, again using the paper as the source but again, not just copying text.
Instead, focus on paraphrasing/synopsising, and extracting the essential
details.}

The problem was solved with the creation of MapReduce - a system that provides
a restricted programming model in order to allow programs in that model to be
automatically distributed with a transparent fault-tolerance system.

Because the idea is so simple, the focus of the paper goes into discussing how
fault-tolerance was achieved, in addition to optimisations and enhancements
that can be made to improve the performance and usability of MapReduce itself.
The innovation comes from the use of the restricted programming model provided.

Fault-tolerance is achieved in two ways: through the use of a distributed
file-system that implements fault-tolerance for input and output data in the
system (although the DFS comes from a separate project) and through the use of
re-executing work that has failed and/or been lost in a way that helps ensure
that the results of such work is unchanged.

\section*{Evaluation}

\textsl{3--4 sentences. How do they evaluate their work? What questions does
their evaluation set out to answer? What does their evaluation say about the
strengths and weaknesses of their system? What is your opinion of the strengths
and weaknesses of the evaluation itself? Give highlights, not a point by point
reproduction of the evaluation section(s). In rare cases, systems papers may
not have any evaluation, in which case write `N/A' below.}

The paper presents an evaluation of the performance of two MapReduce programs
under various conditions. The first is a grep program that searches through one
terabyte of data, and the second is a sorting program that similarly sorts one
terabyte of data. In both cases the data consists of $10^{10}$ 100-byte
records. The evaluation also gives further details with regards to the
configuration of hardware used (on the 1800 machines) and how the input for
both programs was split. In my opinion the evaluation is simple and doesn't
give much meaningful information beyond how MapReduce performs under the very
specific conditions evaluated. Although attempts are made to display its
resilience by killing 200 tasks in one run-through, and to show how having a
\textit{backup task} helps with the tail-end of computation - the results
as-displayed aren't ultimately that convincing. Firstly, because seemingly only
one run-through of each experiment was performed, leading to a lack of
meaningful statistical information. Secondly, the \textit{scaling} properties
of MapReduce aren't really shown, which is inherently one of the most crucial
aspects of a distributed computation framework.

\section*{Your Opinion}

\textsl{At least 3 sentences. This is the fun part where you get to judge both
the paper and the work it reports! Is the motivation convincing? The problem
important? The approach a good or bad idea? Why? Which specific things annoyed
you, or you thought were cool, or cool-but-flawed? Justify your opinions! Make
an argument which will convince others of your opinion.}

Ultimately I think the paper is a good one - in that it presents a powerful and
simple idea and convincingly explains why the idea \textit{could} be powerful
in practise in addition to providing evidence where it \textit{has} had success
at Google. Furthermore, I liked the fact that the paper didn't simply focus on
the core of the idea itself, and also provided lessons learned and extensions
which have proved useful in practise. However, the paper itself does have a few
issues, primarily lying in the evaluation. The lack of even a simple
evaluation, even purely on paper, of the scaling properties of the system is
quite disappointing - and questions on potential bottlenecks, such as the
master node, are left unanswered. Interestingly, Google appears to no longer
use MapReduce internally \cite{NoMoreMapReduce}, although MapReduce (the idea)
has seen continued success in open source systems such as Hadoop \cite{Hadoop}.

\section*{Questions for the Authors}

\textsl{Finally, imagine you're attending a talk about this paper given by one
of the authors. Give at least 2 questions that you would like to ask, specific
to the paper and the research it reports.}

MapReduce is 14 year old now and despite it no longer seeing use internally at
Google, do you foresee other forms of MapReduce lasting or do you think
eventually everyone will move on to other systems?

What is the most interesting feature that you have seen added to a
MapReduce-inspired system that you didn't envisage at the time of publishing
the paper?

\bibliographystyle{unsrt}

\bibliography{references}

\end{document}
