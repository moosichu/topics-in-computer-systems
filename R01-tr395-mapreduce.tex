\documentclass[11pt]{article}

\usepackage{a4wide, parskip}
\usepackage{cite}
\usepackage{hyperref}

\begin{document}

% This work is licensed under the Creative Commons Attribution-NoDerivatives 4.0
% International License. To view a copy of this license, visit
% http://creativecommons.org/licenses/by-nd/4.0/ or send a letter to Creative
% Commons, PO Box 1866, Mountain View, CA 94042, USA.

%% Replace PAPERTITLE below with the paper title
\title{Paper Review Form (1000 words maximum)\\
    tr395: MapReduce: Simplified Data Processing on Large Clusters \cite{MapReduce}}

\maketitle

\section*{Paper Summary}

\textsl{3--5 sentences. Briefly summarise the {\bf contributions} of the paper,
i.e.,~what it adds over the state of the art. Paraphrase and extract the
essentials rather than simply copying chunks of text. Be objective; later
sections allow for your own opinion.}


\section*{Pros and Cons}

\textsl{6 bullets. Succinctly state three positives and three negatives of the
paper.}

Pros:

\begin{itemize}

    \item Pro 1

    \item Pro 2

    \item Pro 3

\end{itemize}

Cons:

\begin{itemize}

    \item IS the master a point of failure?

    \item What if omany of the intermediate keys end up with the same value?

    \item Why use C++ instead of a functional language, so you can guarentee properties like no side-effects.

    \item WHy split map and reduce tasks accross different machines, instead of accross all of them in stages?

    \item The reduce computers read straight from local disk of other machines

    \item if the partitioning function is incorrect, could have man issues.

    \item Getting to run on top of Borg?

    \item What happens if some of the results are read by reduce tasks from failed map macchines. (seems to be handled!)

    \item Why not keep track of if all the reduce tasks have read their data from a map task. What if a reduce task doesn't receive the notification of the change?

    \item What is done about byzantine problems (eg. harddrive corruption?)

    \item Computation aborts if master fail

    \item Does master become a bottleneck?

    \item What are all properties of custom DFS used?

    \item Vast majority of operators are deterministic? Sure deterministic isn't only criteria?

\end{itemize}

\section*{The Problem/Motivation}

\textsl{1--2 sentences per question. What is the motivation for the work, or
the problem being solved? Why is it important? If there is prior art, how was
it insufficient? If the problem had not previously been solved, why not?}



\section*{The Solution/Approach}

\textsl{5--10 sentences. What have they done? How does it address the issues
set out above? How is it unique and/or innovative (if, indeed, it is)? Give
details, again using the paper as the source but again, not just copying text.
Instead, focus on paraphrasing/synopsising, and extracting the essential
details.}


\section*{Evaluation}

\textsl{3--4 sentences. How do they evaluate their work? What questions does
their evaluation set out to answer? What does their evaluation say about the
strengths and weaknesses of their system? What is your opinion of the strengths
and weaknesses of the evaluation itself? Give highlights, not a point by point
reproduction of the evaluation section(s). In rare cases, systems papers may
not have any evaluation, in which case write `N/A' below.}


\section*{Your Opinion}

\textsl{At least 3 sentences. This is the fun part where you get to judge both
the paper and the work it reports! Is the motivation convincing? The problem
important? The approach a good or bad idea? Why? Which specific things annoyed
you, or you thought were cool, or cool-but-flawed? Justify your opinions! Make
an argument which will convince others of your opinion.}
below


\section*{Questions for the Authors}

\textsl{Finally, imagine you're attending a talk about this paper given by one
of the authors. Give at least 2 questions that you would like to ask, specific
to the paper and the research it reports.}

\bibliographystyle{unsrt}

\bibliography{references}

\end{document}
