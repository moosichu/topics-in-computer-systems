\documentclass[11pt]{article}

\usepackage{a4wide, parskip}
\usepackage{cite}
\usepackage{hyperref}

\begin{document}

% This work is licensed under the Creative Commons Attribution-NoDerivatives 4.0
% International License. To view a copy of this license, visit
% http://creativecommons.org/licenses/by-nd/4.0/ or send a letter to Creative
% Commons, PO Box 1866, Mountain View, CA 94042, USA.

%% Replace PAPERTITLE below with the paper title
\title{Paper Review Form (1000 words maximum)\\
    tr395: KLEE: Unassisted and Automatic Generation of High-Coverage Tests for Complex Systems Programs \cite{KLEE}}

\maketitle

\section*{Paper Summary}

\textsl{3--5 sentences. Briefly summarise the {\bf contributions} of the paper,
i.e.,~what it adds over the state of the art. Paraphrase and extract the
essentials rather than simply copying chunks of text. Be objective; later
sections allow for your own opinion.}

The paper introduces KLEE, a symbolic execution tool for automatically
generating tests with the aim of achieving high coverage on a wide variety of
programs. It builds on previous symbolic execution work, applying ideas and
techniques to programs beyond the hand-picked benchmarks that have previously
been used in the field. The contribution over the state of the art is in how
how it aims to tackle the exponential growth in execution paths that can happen
in symbolic execution. Furthermore it aims to solve what is known as the
\textit{environment problem}, where external state can impact the execution of
programs. TODO: eval


\section*{Pros and Cons}

\textsl{6 bullets. Succinctly state three positives and three negatives of the
paper.}

Pros:

\begin{itemize}

    \item Well written, despite no prior experience with the subject matter I
    never once felt confused or lacking for information. Giving a clear usage
    examples helped with this a lot.

    \item Support for LLVM means that KLEE can more easily support many
    languages which have an LLVM backend.

    \item The compact state representation using an immutable heap structure
    is sensible.

    \item The paper lists the optimisations implemented in KLEE that make it
    improve over its predecessor EXE (TODO: cite) \cite{EXE} in a clear and
    concise manner.

\end{itemize}

Cons:

\begin{itemize}

    \item The user is still required to think about what inputs could cause
    errors and tell KLEE which command line argument to use for example.

    \item Although the paper has an explicit evaluation section, some
    evaluation appears in other sections making the paper more cluttered than
    necessary.

    \item The environment modelling claims users need not know about the
    internals of KLEE in order to model the environment. but...

    % Note paper mentions false positives, need to know how many of them exist?

\end{itemize}

\section*{The Problem/Motivation}

\textsl{1--2 sentences per question. What is the motivation for the work, or
the problem being solved? Why is it important? If there is prior art, how was
it insufficient? If the problem had not previously been solved, why not?}



\section*{The Solution/Approach}

\textsl{5--10 sentences. What have they done? How does it address the issues
set out above? How is it unique and/or innovative (if, indeed, it is)? Give
details, again using the paper as the source but again, not just copying text.
Instead, focus on paraphrasing/synopsising, and extracting the essential
details.}


\section*{Evaluation}

\textsl{3--4 sentences. How do they evaluate their work? What questions does
their evaluation set out to answer? What does their evaluation say about the
strengths and weaknesses of their system? What is your opinion of the strengths
and weaknesses of the evaluation itself? Give highlights, not a point by point
reproduction of the evaluation section(s). In rare cases, systems papers may
not have any evaluation, in which case write `N/A' below.}


\section*{Your Opinion}

\textsl{At least 3 sentences. This is the fun part where you get to judge both
the paper and the work it reports! Is the motivation convincing? The problem
important? The approach a good or bad idea? Why? Which specific things annoyed
you, or you thought were cool, or cool-but-flawed? Justify your opinions! Make
an argument which will convince others of your opinion.}
below


\section*{Questions for the Authors}

\textsl{Finally, imagine you're attending a talk about this paper given by one
of the authors. Give at least 2 questions that you would like to ask, specific
to the paper and the research it reports.}

\bibliographystyle{unsrt}

\bibliography{references}

\end{document}
