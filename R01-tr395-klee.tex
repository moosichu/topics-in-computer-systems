\documentclass[11pt]{article}

\usepackage{a4wide, parskip}
\usepackage{cite}
\usepackage{hyperref}

\begin{document}

% This work is licensed under the Creative Commons Attribution-NoDerivatives 4.0
% International License. To view a copy of this license, visit
% http://creativecommons.org/licenses/by-nd/4.0/ or send a letter to Creative
% Commons, PO Box 1866, Mountain View, CA 94042, USA.

%% Replace PAPERTITLE below with the paper title
\title{Paper Review Form (1000 words maximum)\\
    tr395: KLEE: Unassisted and Automatic Generation of High-Coverage Tests for Complex Systems Programs \cite{KLEE}}

\maketitle

\section*{Paper Summary}

\textsl{3--5 sentences. Briefly summarise the {\bf contributions} of the paper,
i.e.,~what it adds over the state of the art. Paraphrase and extract the
essentials rather than simply copying chunks of text. Be objective; later
sections allow for your own opinion.}

The paper introduces KLEE, a symbolic execution tool for automatically
generating tests with the aim of achieving high coverage on a wide variety of
programs. It builds on previous symbolic execution work, applying ideas and
techniques to programs beyond the hand-picked benchmarks that have previously
been used in the field. The contribution over the state of the art is in how
how it aims to tackle the exponential growth in execution paths that can happen
in symbolic execution. Furthermore it aims to solve what is known as the
\textit{environment problem}, where external state can impact the execution of
programs.

\section*{Pros and Cons}

\textsl{6 bullets. Succinctly state three positives and three negatives of the
paper.}

Pros:

\begin{itemize}

    \item Well written, despite no prior experience with the subject matter I
    never once felt confused or lacking for information. Giving a clear usage
    examples helped with this a lot.

    \item Support for LLVM means that KLEE can more easily support many
    languages which have an LLVM backend. (Although this is never mentioned).

    % \item The compact state representation using an immutable heap structure
    % is sensible.

    \item The paper lists the optimisations implemented in KLEE that make it
    improve over its predecessor EXE \cite{EXE} in a clear and concise manner.

\end{itemize}

Cons:

\begin{itemize}

    \item The user is still required to think about what inputs could cause
    errors and tell KLEE which command line arguments to use.

    \item Although the paper has an explicit evaluation section, some
    evaluation appears in other sections making the paper more cluttered than
    necessary.

    \item The paper arguably goes into too much details about specific bugs which
    KLEE found.

    % \item The environment modelling claims users need not know about the
    % internals of KLEE in order to model the environment. but...

    % Note paper mentions false positives, need to know how many of them exist?

    % \item Relies on heavily on applications to be deterministic, which is an
    % issue with multi-threaded systems.

    % \item There are some issues with the KLEE's environment system. Such
    % as the assumption that the results of a concrete function call will be
    % deterministic. % Solution to this?

    % \item Implementing the environment at the system call level instead
    % of the C library level is less portable.

    % \item Using a custom version of the C standard library (instead of the
    % default one), and checking \textit{it} for errors further reduces
    % portability.

    % \item Is there anything someone can do to mark library code as out of their
    % control. (And therefore potentially even have their code avoid the erronous
    % code?).

    % \item Can users state custom end-state failure cases?

    % \item Inconsistenty in example calls to KLEE (./run vs klee).

    % \item Is there a way to test non-batch programs? Function calls?

    % \item Is how long does it take to run the tests KLEE generates vs manual
    % tests? Is testing relative coverage against manual tests fair because
    % normally these test for regressive behaviour? Taking 89 hours to run is a
    % long time. Time of length of execution doesn't say much without hardware it
    % ran on. Completely different types of tests.

    % \item Asserts take advantage of it well! (essentially depends on inline for
    % code coverage meaningful). Concrete vs abstract test cases.

    % \item Arguably goes too much into specific bugs that were found, strese the
    % point a lot.

    % \item Code coverage is a weird heuristic to test against. Does KLEE generate
    % tests to test for code not covered?

    % \item One of the footnotes is cutoff (page 220).

    % \item Can it integrate into version control systems to check which lines of
    % code have been changed to make. Automated regression tests? "Bake"
    % coverage?

\end{itemize}

\section*{The Problem/Motivation}

\textsl{1--2 sentences per question. What is the motivation for the work, or
the problem being solved? Why is it important? If there is prior art, how was
it insufficient? If the problem had not previously been solved, why not?}

Computer programs are buggy \cite{GitHubBugs}, however, finding those bugs is
hard \cite{FindingBugsHard}. Therefore, tools to find potential bugs in
programs is an important problem. Symbolic execution tools aim to help with one
particular aspect of finding erroneous code, by being able to execute many
possible execution paths in a program in order to find statements where an
undesired state \textit{could} be reached given valid inputs. However, previous
tools struggled to operate on larger programs (due to exponential path blow-up)
and on programs which interact with an external environment (such as an
operating system).

\section*{The Solution/Approach}

\textsl{5--10 sentences. What have they done? How does it address the issues
set out above? How is it unique and/or innovative (if, indeed, it is)? Give
details, again using the paper as the source but again, not just copying text.
Instead, focus on paraphrasing/synopsising, and extracting the essential
details.}

The authors of the paper have created a symbolic execution tool called KLEE,
which is built on top of previous work by the authors \cite{EXE}. It improves
upon EXE in two ways. The first, is in the implementation of a variety of
optimisation techniques to help mitigate the path-explosion problem, meaning
far fewer states are created when KLEE symbolically executes a program. The
second improvement is the introduction of a mechanism to help tackle the
problem of symbolically executing a program's interaction with an external
environment. This is unique and innovative in allowing a user to implement
their own version of system-calls which can be symbolically executed by KLEE.
As this can be done in the same language as the user's programs, they could
potentially write their own custom implementations of these.

\section*{Evaluation}

\textsl{3--4 sentences. How do they evaluate their work? What questions does
their evaluation set out to answer? What does their evaluation say about the
strengths and weaknesses of their system? What is your opinion of the strengths
and weaknesses of the evaluation itself? Give highlights, not a point by point
reproduction of the evaluation section(s). In rare cases, systems papers may
not have any evaluation, in which case write `N/A' below.}

The evaluation mainly takes the form of evaluating the code coverage KLEE
manages to achieve on a selection of programs, including command line tools and
an operating system kernel. The questions it aims to answer are: \textit{How
does the code coverage of KLEE compare with a hand-written test-suite? Is KLEE
able to produce meaningful results?} The evaluation demonstrates two strengths
of the system. Firstly, previous symbolic code execution tools don't appear to
have been able to execute programs of the scale KLEE can handle. Secondly, KLEE
does manage to achieve comprehensive code coverage, often being able to achieve
a code coverage of 90\% for the programs tested. The \textit{proof in the
pudding} comes from the fact that these tests did produce meaningful results,
and managed to uncover real bugs in some widely used programs such as GNU
Coreutils \cite{Coreutils}.

Although the evaluation demonstrates the perceived strengths of KLEE, on of its
pitfalls comes through from not discussing or exploring its potential
weaknesses. For example, no mention is made on the type of programs KLEE could
struggle with, or what could cause it to achieve poor code coverage.
Furthermore, no mention is made of the running-time beyond stating that one of
the runs of KLEE took 89 hours.

Furthermore, comparing the code coverage of a symbolic execution system with
hand-written unit-tests seems like a flawed comparison, as they test for
different things and KLEE would be an addition, not a replacement for them.

\section*{Your Opinion}

\textsl{At least 3 sentences. This is the fun part where you get to judge both
the paper and the work it reports! Is the motivation convincing? The problem
important? The approach a good or bad idea? Why? Which specific things annoyed
you, or you thought were cool, or cool-but-flawed? Justify your opinions! Make
an argument which will convince others of your opinion.}

Overall I think the paper and the work it reports is good. I find the
motivation convincing, making bugs in programs easier to find is always a goal
worth striving for! The most impressive aspect of the paper is the fact that
the system was used on GNU Coreutils and managed to find a few undetected bugs.

The paper does have a few flaws. The paper avoids mention of any potential
further work or improvements to the system, making it hard for the reader to
know if potential shortfalls could be address. For example, KLEE is unable
non-deterministic programs, but no mention is made of how that could be tested
for or even worked around.

\section*{Questions for the Authors}

\textsl{Finally, imagine you're attending a talk about this paper given by one
of the authors. Give at least 2 questions that you would like to ask, specific
to the paper and the research it reports.}

Although not done for this implementation (due to the complexity of the
problem), would you ideally mock C-library calls instead of system calls?

Would it be possible to implement optimisations which work in conjuction with
version control systems to ensure that only changes to programs get tested to
help cut down on the run-time of KLEE?

\bibliographystyle{unsrt}

\bibliography{references}

\end{document}
